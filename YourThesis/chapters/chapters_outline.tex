% !TEX root=../../Thesis.tex
\chapter{Modeling Intersection Driving Scenarios}
\label{ch:modeling_intersection}
\section{Intersection scenarios}
\section{State space}
\section{Action space}
\section{Transistion model}
\subsection{Simulation scenarios or simulator}
\section{Observation model}
\section{Reward function}
\section{Discussion}




\chapter{Learning without a transition model}
\tommy{Deep Q-learning approach}
We want to formulate the problem in such a way that abstracts the information of traffic lights, traffic signs and intention. This way the car is closer to L5 by not relying on the different traffic lights

One motivation example is in how traffic lights works. In sweden, we have sensors that can sense if there are cars in an intersection and create a traffic light schedule accordingly, compared to the US where the traffic signals set up using timers. As a consequense, Drivers approacing a yellow light 

\section{Approach (State represenation, observable and unobservable)}
raindow paper, combine a number of state of the art technics and show what works 
\todo{State representation}
This section describe the general state representation used in this research that enables these methods to be generalizable for different type of intersection and crossings. 
By describing the state space as a set of distances to intersection points, we can abstract the map layout of different intersections and the same algorithms would work for intersections variations that we haven't specifically trained on. 
\todo{add image of intersection scenario with 90 degree entry point and 45 degree entry point.}
\todo{Observable states}
Position, velocity and acceleration. 
\todo{Unobservable states}
Intentions. Traffic lights and traffic signs. 
\section{Simulated experiments}
\section{Results and discussion}

\chapter{Combining reinforcement learning and model based optimization}
\todo{as we see in previous chapter }
\section{MPC}

\section{Approach, (Action space, options. MPC)}
MPC has a mixed integer problem, calculating the optimal path for all possible action is very computationally heavy. 
RL DQN. Only has discrete actions. Can not guarantee safety but is good at choosing actions with the best utility (value). 
The reward function takes in the predicted outcome of the model in the MPC and can penalize the choice of action. but if experience show that the outcome is better than the model, it can choose to take a bad action that would lead to a better total reward compared to only following a conservative model. 


\section{Simulated experiments}
We show the difference in a multi crossing scenario where the MPC can plan a path for both intersections while our previous DQN only handles one at a time. 

\section{Results and discussion}

\chapter{Estimating the uncertainty}
\todo{Rewrite chapter name}

\section{Uncertainty of the decision}
\tommy{NN is a black box. The utility value q that come from the DQN works great if it was trained on the }
\subsection{Approach}
\subsection{Simulated experiments}
\subsection{Results and discussion}

\section{Uncertainty of the intention}
\subsection{Approach}
\subsection{Simulated experiments}
\subsection{Results and discussion}


\chapter{Generalizing over different scenarios}
\tommy{Uncertainty in MDP which one are we in?}
As mentioned in the seciton \ref{mdp}. the \gls{mdp} is defined by the tuple (S,A,R,T,$\gamma$). the work until now has solved the \gls{mdp} or \gls{pomdp} without defining the transition function T thanks to \gls{rl}. So what happens when you take a policy trained on one \gls{mdp} defined by one transition function? Well the short answer is that because \gls{dqn} uses a \gls{nn} to approximate the utility Q for taking an action in each state. 

Application areas. A transition function is defined as the probability of transitioning from one state to another. This could be different for example when the ego vehicle properties are different, if the DQN is trained on a sports car and then applied on a truck, the difference in acceleration capability would result in a different transition function. Another example would be on the other traffic participants, if the DQN is train in an environment where the driving culture is on the passive side and that is later put in an environment that has a more aggressive driving culture. The DQN/policy would probably not perform so well. 
This is especially important when you have an intention state directly correlated with the transition. For example in \paperD were the intentions are described on a high level such as take way or give way. 
An example is lane changes in Sweden, it is normal to signal first, wait for the other vehicle to slow down before initiating a lane change. While in a high density traffic jam in Paris, it is more normal to show intention by starting a lane change and observe if the other vehicle yeilds. 

\section{Approach}
Some easy solutions would to be have some geo identifier that can choose which policy to use given the country. That may work for an L4 system but could show to be difficult for an L5 system. Our approach is to use transfer RL to identify where we are in the convex hull of MDPs and then choose the policy for MDP we are closest too. 
This does require some number of MDPs to span out the convex hull of models. 
This can find a policy between MDPs. 
Identify which MDP is closest. For example Downtown driving in one country may be similar to the driving stile to another. 
\todo{cons: computationally heavy. Have to create the complex hull of MDPs from e set of MDPs.}
\todo{Pros: able to identify which policy to use and scale better by generalizing MDPs instead of countries.}


\section{Simulated experiments}
\section{Results and discussion}
FILL