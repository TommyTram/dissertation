\chapter{Estimating the uncertainty}
\todo{Rewrite chapter name}
The motivation of handling uncertainty. In this chapter we present two approches to handling the uncertainty, one is the uncertainty in output of the \gls{dqn} and the other in the uncertainty of the intention estimation that is feed as an input to the \gls{dqn}.
\section{Uncertainty of the decision}
\tommy{NN is a black box. The utility value q that come from the DQN works great if it was trained on the }
\subsection{Approach}
\subsection{Simulated experiments}
\subsection{Results and discussion}

\section{Uncertainty of the intention}
In paper A the policy put itself in a position that would not be in conflict with another cars time to intersection and could avoid a lot of the collisions. But the cases the cars collided was when in somehow ended up in a collision course and thats when it had trouble making its way out. 

\subsection{Approach}
We create a belief state using the probability distribution of estimated intention and compare four different methods of handling that 
\subsection{Simulated experiments}
\subsection{Results and discussion}