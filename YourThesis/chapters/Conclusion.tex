\chapter{Concluding Remarks and Future Work}
\label{sec:conclusion}
%
% Q1. What requirements need to be set on the sensor-suite and prediction
% algorithms in order to enable safe autonomous driving?
% Q2. How should a vehicle controller be designed in order for safety to be
% proven by design?
% Q3. Can a safe AD framework be deployed on a real test platform?

While trying to answer the three research questions presented in Section~\ref{chapter:intro}, this thesis introduced a set of different MPC formulations intended for AD applications. In particular, we introduced MPC in a general sense in Chapter~2, and discussed why the standard results available in the literature cannot be directly applied to ensure safety in AD applications.  

Indeed, in order for a self-driving vehicle to become safe, it must be able to sense and predict its surrounding environment. Chapter~3 therefore introduced an efficient pedestrian prediction model (\paperPedestrian{}) and specified a requirement on prediction properties (Assumption~\ref{a:unknown_constraints}) that proves to be fundamental in ensuring safety (research question Q1).

Assuming that the environmental prediction models have a specific structure, we then show that safety guarantees could be enforced by the introduction of a safe set (\paperSafe). In other words, designing an MPC controller that is safe by design is possible (research question Q2). However, since MPC is an optimization-based technique, and can in general be computationally demanding, we show in Chapter~4 that it is still possible to deploy a safe MPC framework in a real vehicle platform (\paperPlanner, \paperExp), while satisfying the safety guarantees in Theorem~\ref{theorem:safe} (research question Q3).


Even though this thesis presented an approach to ensure safety for AD applications, it is far from being a ``silver bullet'' that solves all problems related to autonomous driving. Therefore, we will next comment on some possible extensions and also outline future research directions.



\section{Future Work}
While the results in Theorem~\ref{theorem:safe} guarantee safety, i.e., recursive feasibility, of the controller, it does not prove asymptotic stability when the a-priori unknown constraints are active. A promising future research direction, can therefore be to first expand the ISS results from \paperISS{} to general nonlinear systems, and then show some sort of ISS results also for Theorem~\ref{theorem:safe} by considering the road users as an input. Furthermore, while \paperISS{} showed that using infeasible references can still yield some stability results, it may be quite limiting to always use a pre-defined reference trajectory. To that end, it would also be interesting to combine the proposed framework with some form of re-planning or online learning, e.g., updating the reference velocity in the presence of road users that make the pre-defined reference infeasible.


\paperScenario{} showed that multi-modal prediction models could be used to improve the overall performance of the system. However, the results relied partly on the availability of such models, but also on a probability estimate for each predicted mode. Indeed, to be able to use these results more widely, a future direction would be to use some learning-based methods to estimate such probabilities from real data. In addition, since the predictability of the environment plays a large part in enabling safe autonomous driving applications, it is necessary to also direct research attention towards deriving models that can accurately represent the surrounding environment.


Finally, in order to further strengthen the results presented in this thesis, future work aims to implement the framework from \paperExp{} to a more general autonomous driving setting, where road users are not simulated, but measured and predicted in real-time.






%The predictability of the environment plays a large part in making safe autonomous driving applications a possibility. Indeed, being able to represent the surrounding environment in the least conservative, but most accurate way will enable the self-driving technology to progress even further. However, in order to so, some additional rules may need to be more clearly defined. For instance, prediction models that consider all worst-case behaviors will force a self-driving vehicle to drive slower than any human driver would, e.g., a conservative model that predicts that every pedestrian will intentionally run onto the road at all times will greatly limit the velocity at which the vehicle can drive, while always being able to capture the worst outcome. On the other hand, a model that predicts behaviors based on rules following from the legal road code may not model all pedestrians accurately. As such, this may require one to reason about a trade-off between performance and a willingness to take risks. Hence, new frameworks that ensure safety w.r.t a pre-specified risk-level may need to be considered instead.

%Future work aims to implement the safe MPC framework presented in Paper F in more complicated settings with actual road users and to verify the performance for more complicated settings.

% drive more cautiously than models that predict some nominal behavior that abides by a road code. However, it is clear that such a model



% For instance, one of the biggest 


% One of the most important aspects when it comes to 


% This thesis has discussed the importance of having consistent prediction models of the surrounding environment. 


% Concluding remarks
% % Lorem ipsum dolor sit amet, consectetur adipiscing elit, sed do eiusmod tempor incididunt ut labore et dolore magna aliqua. Ut enim ad minim veniam, quis nostrud exercitation ullamco laboris nisi ut aliquip ex ea commodo consequat. Duis aute irure dolor in reprehenderit in voluptate velit esse cillum dolore eu fugiat nulla pariatur. Excepteur sint occaecat cupidatat non proident, sunt in culpa qui officia deserunt mollit anim id est laborum.

% Lorem ipsum dolor sit amet, consectetur adipiscing elit, sed do eiusmod tempor incididunt ut labore et dolore magna aliqua. Ut enim ad minim veniam, quis nostrud exercitation ullamco laboris nisi ut aliquip ex ea commodo consequat. Duis aute irure dolor in reprehenderit in voluptate velit esse cillum dolore eu fugiat nulla pariatur. Excepteur sint occaecat cupidatat non proident, sunt in culpa qui officia deserunt mollit anim id est laborum.

% Lorem ipsum dolor sit amet, consectetur adipiscing elit, sed do eiusmod tempor incididunt ut labore et dolore magna aliqua. Ut enim ad minim veniam, quis nostrud exercitation ullamco laboris nisi ut aliquip ex ea commodo consequat. Duis aute irure dolor in reprehenderit in voluptate velit esse cillum dolore eu fugiat nulla pariatur. Excepteur sint occaecat cupidatat non proident, sunt in culpa qui officia deserunt mollit anim id est laborum.Lorem ipsum dolor sit amet, consectetur adipiscing elit, sed do eiusmod tempor incididunt ut labore et dolore magna aliqua. Ut enim ad minim veniam, quis nostrud exercitation ullamco laboris nisi ut aliquip ex ea commodo consequat. Duis aute irure dolor in reprehenderit in voluptate velit esse cillum dolore eu fugiat nulla pariatur. Excepteur sint occaecat cupidatat non proident, sunt in culpa qui officia deserunt mollit anim id est laborum.