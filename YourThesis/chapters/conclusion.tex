% !TEX root=../../Thesis.tex
\chapter{Concluding remarks and future work}
\section{Conclusions}\label{ch:conclusion}
Looking at the reserach questions we posted in the beginning: 
\begin{enumerate}
	\item[\textbf{Q1.}] How can \gls{rl} techniques be used to develop a decision-making agent that effectively navigates intersections without explicitly estimating the intention state of other vehicles? 
	\item[\textbf{A1.}]By formulating the problem as a \gls{pomdp} with an unobservable intention state. Deep Q-learning showed potential in developing effective policies. Using short-term goals as actions and methods like IDM or MPC to control these actions for navigating intersections was proven to be effective.
	\item[\textbf{Q2.}] How can an \gls{rl} agent utilize the uncertainty in its predictions and actions to enhance decision-making in complex environments? 
	\item[\textbf{A2.}] By creating estimate of the uncertainty of the Q-value from a \gls{dqn} a threshold value was able to reduce the number of simulated collisions. Similarly, another threshold  value can be applied to the estimation of the intention state, directly influencing the risk level taken by the driving policy. These approaches leverage uncertainty estimation to make more informed decisions in challenging and dynamic environments.
	\item[\textbf{Q3.}] How do can a \gls{rl} agent handle situations it has not been trained on? 
	\item[\textbf{A3.}] If we have a set of pretrained \gls{dqn} models, transfer learning can be employed to switch models whenever it is identified that the agent is in an environment it has not been trained on. This allows the agent to adapt more effectively to new and unfamiliar scenarios by leveraging knowledge from previous training experiences.
\end{enumerate}



% 1. RL by itself still has a long way to guarantee safety, but the methods presented in this paper. The uncertainty can be reduced. Safety is better suited for control or formal methods. RL is a great tool for creating policy that can adapt to different driver intentions. 
% 2. Even with todays advancements in \gls{nn} DQN still has a hard time handling 
\section{Future work}
Future research should focus on transitioning from simulation environments to real-world implementations. While simulations offer a controlled setting for developing and testing algorithms, real-world driving presents unpredictable variables and complexities. Implementing and refining these models in actual driving scenarios will be crucial for validating their effectiveness and reliability. This step will involve rigorous testing, continuous learning, and adaptation to ensure the autonomous systems can handle diverse and dynamic real-world conditions, ultimately moving closer to the widespread adoption of safe and efficient autonomous vehicles.

Additionally, the integration of language prediction models, specifically using transformers and attention mechanisms, along with driver monitoring systems (DMS), can enhance the prediction of driver intentions. Transformers, with their powerful attention mechanisms, can effectively handle sequential data and capture complex dependencies. By leveraging these models, it may be possible to interpret and predict driver behaviors based on a broader range of contextual cues.

Driver monitoring systems can provide critical real-time data on driver behavior, including eye movement, head position, and other physiological indicators. Combining this data with sensor inputs and verbal communication or textual descriptions of driver actions can create a comprehensive understanding of driver intentions. Transformers can process and correlate these different data types, improving the accuracy of intention estimation, particularly in complex or ambiguous driving scenarios.

Moreover, employing transformers trained on extensive datasets could facilitate the development of more sophisticated algorithms capable of predicting and adapting to diverse driving behaviors. This approach could significantly enhance the overall safety and efficiency of autonomous driving systems by providing a more nuanced and dynamic understanding of the driving environment. Integrating DMS with advanced language models could also help in identifying potential risks and ensuring timely interventions, thus improving the overall robustness and reliability of autonomous vehicles.