\documentclass{standalone}
\usepackage[mode=buildnew]{standalone}
\usepackage{tikz}
\usepackage{amsmath,array}
\newcommand{\dataset}{{\cal D}}
\newcommand{\fracpartial}[2]{\frac{\partial #1}{\partial  #2}}
\newcommand{\RNum}[1]{\uppercase\expandafter{\romannumeral #1\relax}}
\usepackage{mathtools}
\usepackage{amssymb}
\usepackage{bm}
\usetikzlibrary{positioning, shapes, arrows}
\begin{document}
\begin{tikzpicture}[thick,scale=5]
\coordinate (A1) at (5.0,0.2);
\coordinate (A2) at (4.8,0);
\coordinate (A3) at (4.8,-0.3);
\coordinate (A4) at (5.1,-0.5);
\coordinate (A5) at (5.4, -0.15);

\node (mu1) at (5.0, 0.25) {$\mu_1$};
\node (mu2) at (4.75, 0) {$\mu_2$};
\node (mu3) at (4.75, -0.3) {$\mu_3$};
\node (mu4) at (5.1, -0.55) {$\mu_4$};
\node[text=red] (mu5) at (5.45, -0.15) {$\mu_5$};



\draw (A1) -- (A2) -- (A3) -- (A4) -- (A5) -- cycle ;
\path (A1.south) |- (A5.west) node [midway,above] {$\mathcal{C}(\mathcal{M}_s)$};

\node (type5) at (5.9, -0.5) [circle,fill,color=violet,inner sep=1pt]{};
\node (type1) at (5.4, -0.15) [circle,fill,color=red,inner sep=1pt]{};
\node (type3) at (5.1, -0.35) [circle,fill,color=blue,inner sep=1pt]{};

\matrix [draw,below left] at (current bounding box.north east) {
  \node [circle,fill,color=red,inner sep=1pt, type1,label=right:Type I] {}; \\
  \node [circle,fill,color=blue,inner sep=1pt, type3,label=right:Type III] {}; \\
  \node [circle,fill,color=violet,inner sep=1pt, type5,label=right:Type VI] {}; \\
};

% or a cleaner version, as @TorbjørnT. said in the comments:
% \node at (A1.south |- A5.west) {$\pi$};

\end{tikzpicture}
\end{document}