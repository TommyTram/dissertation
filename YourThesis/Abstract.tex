% !TeX spellcheck = <none>
The main topic for this thesis is tactical decision making for autonomous driving through intersections with other road users. A human driver are able to drive in diverse environments and situations even though they have never driven there before. The same is expexed of an \gls{av}. Specifically, this thesis study the problem of navigating though an intersection where the intention of other drivers are unknown. These intentions depend on a veriaty of factors such as the mood or attentionf of the driver, right of way, traffic signs or lights. By generalizing the future action of the driver to intention, the autonomous vehicle will be able to handle more complicated scenarios such as a driver running a red light. 
The large amount of environments and possible scenarios makes it hard to manually specify a reaction for every possible situation
Therefore, a learning-based strategy is considred in this thesis, which will introduce different approaches based on \gls{rl}. 

The problem is formulated as a \gls{pomdp} to account for the unknown intentions and a genereal decision making agent derived from the \gls{dqn} algorithm is proposed. With few modifications, this method can be applied to different environments, which is demonstrated for various simulated intersection scenarios. 

% Autonomous driving technologies have been developed in the past decades with the objective of increasing safety and efficiency. However, in order to enable such systems to be deployed on a global scale, the problems and concerns regarding safety must be addressed. The difficulty in providing safety guarantees for autonomous driving applications comes from the fact that the self-driving vehicle needs to be able to handle a diverse set of environments and traffic situations. More specifically, it must be able to interact with other road users, whose intentions cannot be perfectly known.


% This thesis proposes a Model Predictive Control~(MPC) approach to ensure safe autonomous driving in uncertain environments. While MPC has been widely used in motion planning and control for autonomous driving applications, the standard literature cannot be directly applied to ensure safety (recursive feasibility) in the presence of other road users, i.e., pedestrians, cyclists, and other vehicles. To that end, this thesis shows how recursive feasibility can still be obtained through a slight modification of the MPC controller design.


% The results of this thesis build upon the assumption that the behavior of the surrounding environment can be predicted to some extent, i.e., a future motion trajectory with some uncertainty bound can be propagated. Then, by postulating the existence of a safe set for the autonomous driving problem, and requiring that the motion prediction models have a consistent structure, safety guarantees can be derived for an MPC controller.

% Finally, this thesis shows that the proposed MPC framework does not only hold in theory and simulations, but that it can also be deployed on a real vehicle test platform and operate in real-time, while still ensuring that the conditions needed for the derived safety guarantees hold.