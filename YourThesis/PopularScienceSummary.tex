Our world of transportation is rapidly evolving, with autonomous driving (AD) technology set to revolutionize urban mobility by redefining how we navigate cities, enhancing safety, and paving the way for a more efficient, inclusive, and sustainable future. However, navigating intersections safely remains a critical challenge on the path to this future. This thesis explores using Deep Q-learning to train decision-making agents for intersection navigation, focusing on understanding and managing uncertainty in other drivers' intentions.
Deep Q-learning is a Reinforcement Learning (RL) algorithm which learns through trial and error. In this thesis, agents were developed and tested to master intersection strategies within simulated environments. The goal is to equip these agents with the ability to make split-second decisions based on real-time data about vehicle positions and velocities to estimate what other drivers might do next.
\\

Key highlights of the research include the successful integration of RL algorithms with advanced control methods like Model Predictive Control (MPC), addressing uncertainties through sophisticated AI techniques, and leveraging previous models to train new ones. This combined approach significantly enhanced the agents' ability to anticipate and respond to unpredictable driver behaviors, contributing to the advancement of safety and reliability in autonomous driving systems within real-world environments. The findings of this thesis hold promise for future innovations in AI-driven transportation systems, aiming towards safer roads and more efficient traffic management solutions in our increasingly interconnected world.

%  to navigate intersections in dynamic traffic scenarios.
% By considering the uncertainty of driver intentions enhanced the agents' ability to anticipate and respond to unpredictable driver behaviors, this research contributes to advancing the safety and reliability of autonomous driving systems in real-world environments.
% The findings of this thesis hold promise for future innovations in AI-driven transportation systems, promising safer roads and more efficient traffic management solutions in our increasingly interconnected world.

% Imagine a future where autonomous vehicles (AVs) seamlessly optimize traffic flow, reducing congestion and revolutionizing urban mobility. However, before we can reach that future navigating intersections safely remains a critical challenge. This thesis dives into this issue by exploring how advanced machine learning techniques, specifically reinforcement learning (RL), can be harnessed to train decision-making agents capable of safely maneuvering through complex intersections. The focus is not just on driving skills but also on understanding and managing the uncertainty surrounding the intentions of other drivers.
% Using RL, which learn through trial and error, various RL algorithms was developed and tested to teach agents the optimal strategies for intersection navigation in a simulation environment. The goal was to equip these agents with the ability to make split-second decisions based on real-time data about vehicle positions, velocities, and heading, to estimate what other drivers might do next.

%By intelligently scheduling commercial transports outside of peak hours, AVs have the potential to significantly enhance traffic efficiency. Moreover, they could alleviate the strain on parking infrastructure by autonomously relocating to less crowded areas when not in use.

% Machine learning (ML) has been instrumental in driving the rapid progress of AD technology. Advanced ML techniques enable AVs to perceive their surroundings with unprecedented accuracy. However, navigating complex scenarios such as busy intersections remains a challenge, as it requires intricate interaction with other road users. Human drivers possess the ability to assess the environment, identify traffic signals, and make split-second decisions based on various factors. Yet, human error still leads to numerous accidents annually, underscoring the need for more sophisticated decision-making algorithms for AVs.

% The focus of this thesis revolves around the development of decision-making algorithms that go beyond merely adhering to traffic rules, but also anticipate and adapt to the actions of other drivers. Leveraging reinforcement learning (RL) methods, this thesis proposes the creation of a decision-making agent capable of navigating intersections while taking into account the intentions of other drivers. Additionally, two approaches to estimating uncertainty are proposed. This uncertainty estimate serves as a crucial tool for avoiding poor decisions and ensuring the implementation of a safer driving strategy.