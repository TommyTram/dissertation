Our means of transportation are undergoing a remarkable evolution, with autonomous driving (AD) technology poised to play a pivotal role in this transformation. Imagine a future where autonomous vehicles (AVs) seamlessly optimize traffic flow, reducing congestion and revolutionizing urban mobility. By intelligently scheduling commercial transports outside of peak hours, AVs have the potential to significantly enhance traffic efficiency. Moreover, they could alleviate the strain on parking infrastructure by autonomously relocating to less crowded areas when not in use.

Machine learning (ML) has been instrumental in driving the rapid progress of AV technology. Advanced ML techniques enable AVs to perceive their surroundings with unprecedented accuracy. However, navigating complex scenarios such as busy intersections remains a challenge, as it requires intricate interaction with other road users. Human drivers possess the ability to assess the environment, identify traffic signals, and make split-second decisions based on various factors. Yet, human error still leads to numerous accidents annually, underscoring the need for more sophisticated decision-making algorithms for AVs.

The focus of this thesis revolves around the development of decision-making algorithms that go beyond merely adhering to traffic rules, but also anticipate and adapt to the actions of other drivers. Leveraging reinforcement learning (RL) methods, this thesis proposes the creation of a decision-making agent capable of navigating intersections while taking into account the intentions of other drivers. Additionally, two approaches to estimating uncertainty are proposed. This uncertainty estimate serves as a crucial tool for avoiding poor decisions and ensuring the implementation of a safer driving strategy.